\documentclass[12pt]{article}

\usepackage[margin=1.1in]{geometry}
\input{../../syllabi/preamble}

%%edit these!!
\newcommand{\coursedept}{MATH}
\newcommand{\coursenumber}{243}
\newcommand{\professorname}{Adam Kapelner, Ph.D.}
\newcommand{\sixhundredsection}{}
\newcommand{\coursenumbercrosslisted}{}
\newcommand{\semester}{Fall}
\newcommand{\numcredits}{4}
\newcommand{\lectimeandloc}{Tues and Thurs 5-6:50PM}
\newcommand{\professorcontactinfo}{Write to \texttt{@kapelner} on \slacklink}
\newcommand{\professoroffice}{604 Kiely Hall}
\newcommand{\requiredlabtimeandloc}{}
\newcommand{\tataofficehourtimeandloc}{} %TA / TA Office Hours / Loc & Kennly Weerasinghe / see \coursewebpagelink
\newcommand{\numtheoryhws}{6--9}
\newcommand{\lastdatetimetohandinhomeworks}{Dec 15 at noon}
\newcommand{\midtermonedateandlocation}{Thursday, October 7 on zoom during class time}
\newcommand{\midtermtwodateandlocation}{Thursday, November 11 on zoom during class time}
\newcommand{\finaldateandlocation}{TBD but on zoom}

%%%unchanged
\newcommand{\coursewebpageurl}{https://github.com/kapelner/QC_\coursedept_\coursenumber_\semester_\the\year}
\newcommand{\coursewebpagelink}{\href{\coursewebpageurl}{course homepage}}
\newcommand{\slackurl}{https://QC\coursedept\coursenumber\semester\the\year.slack.com/}
\newcommand{\slacklink}{\href{\slackurl}{slack}}

\input{../../syllabi/_header}



\section*{Course Overview}

\coursedept~\coursenumber~is an introduction to the basic concepts and techniques of probability, statistics and statistical computation with an emphasis on applications. Topics to be covered are below (not in order of coverage):

\begin{itemize}
\itemsep -0.0em 
\item Basic Set Theory
\item Counting Methods --- permutations and combinations
\item Basic Probability Theory --- axioms, conditional probability, in/dependence, iid
\item Modeling with Discrete Random Variables: Bernoulli, Hypergeometric, Binomial, Poisson, Geometric, Negative Binomial, Uniform Discrete, Rademacher and others
\item Support, Parameters, parameter space, expectation, variance, moments
\item Modeling with Continuous Random Variables: Exponential, Uniform and Normal
\item Covariance and correlation
\item Moment Generating Functions
\item Law of Large Numbers and the Central Limit Theorem
\item Concepts of populations, samples, sampling, surveys, representativeness, generalizability
\item The three goals of statistical inference: point estimation, theory testing (with alpha level) and confidence set construction. 
\item Definition of statistic, estimator, estimate. 
\item For the Bernoulli data generating process with parameter $\theta$, point estimation, confidence intervals and hypothesis testing (left, right and two-sided one-proportion z-tests and their rejection and retainment regions)
\item Fisher's $p$-value, statistical significance, practical significance
\item Type I and Type II errors, power calculations for the proportion
\item Point estimation, confidence intervals and hypothesis testing for one-sample proportions
\end{itemize}

Students taking this course may not receive credit for MATH 114 or MATH 241 except by permission of the chair. Corequisites include MATH 120, 142 or 151 and CSCI 111. \textbf{This is not your typical mathematics course.} This course develops ideas and concepts for helping to make decisions based on randomness and we will do lots of modeling of real-world situations. The course does not dwell on theory nor details of computation but will make use of computation especially using the \texttt{Python} language.


\section*{Course Materials}

\paragraph{Textbook:} A First Course in Probability by Sheldon Ross. I prefer the 7th edition which you can buy this used on \href{http://www.amazon.com/First-Course-Probability-7th-Edition/dp/0131856626}{Amazon} for cheap. You can buy \textit{any edition} though if you find it cheaper. There is no excuse not to have this book. It is \textit{required}. However, I will not be teaching \qu{from the book} and most of the material in the class comes from the lecture notes. The textbook is a way to get ``another take'' on the material.

\paragraph{Computer Software:} We will also be using \texttt{Python} which is a free, open source programming language and console. To download \texttt{Python} go to \url{https://www.python.org/downloads/}. I then recommend the IDE \texttt{Jupyter Notebook} available for free so you can follow Amir's Python translations of the class demos. To install Notebook, open up a command prompt and executing \texttt{pip install notebook}. After this is installed, you can run \texttt{jupyter notebook} in a command prompty which will open up a web browser with the IDE embedded. An alternative method is to install Anaconda using \href{https://medium.com/@yunanwu2020/how-to-install-jupyter-notebook-from-scratch-on-anaconda-simple-codes-8652c3095091}{these instructions}.

\paragraph{Calculator:} You can use a TI-84, 85, 89 or any calculator which you wish. I strongly suggest you use \href{http://www.wolframalpha.com/}{Wolfram Alpha} and its smartphone app.


%\input{../../syllabi/_the650section}

\input{../../syllabi/_announcements_on_slack}

\input{../../syllabi/_use_of_slack}

\input{../../syllabi/_standard_class_meetings}

%\input{../../syllabi/_jewish_holiday_reschedule}

%\input{../../syllabi/_zoom_policies}

%\input{../../syllabi/_lecture_upload}

\section*{Homework}

\input{../../syllabi/_theory_hws_text}

\input{../../syllabi/_theory_hws_submission_text}
\input{../../syllabi/_philosophy_hws}

\input{../../syllabi/_time_spent_hws}

\input{../../syllabi/_late_hw_policy}

\input{../../syllabi/_latex_hw_bonus_policy}

\input{../../syllabi/_hw_ec_policy}

\section*{Examinations}

\input{../../syllabi/_examination_text}

\input{../../syllabi/_standard_exam_schedule}

\subsection*{Exam Policies and Materials}

\input{../../syllabi/_examination_policies}

%\input{../../syllabi/_zoom_examination_policies}

\input{../../syllabi/_standard_cheat_sheet_policy}


\input{../../syllabi/_cheating_on_exams_and_missing_exams}
\input{../../syllabi/_special_services}

\input{../../syllabi/_class_participation}

%\input{../../syllabi/_zoom_attendance}

\input{../../syllabi/_standard_grading_and_grading_policy}

\input{../../syllabi/_241_grade_distribution}

Note: 243 is a new course so this section was written based on 241 courses.


%\input{../../syllabi/_advanced_course_grade_distribution}

%\input{../../syllabi/_241_cs_c_minus_notice}

\input{../../syllabi/_grade_checking_on_gradesly}

\input{../../syllabi/_auditing_policy}

\pagebreak
\input{../../syllabi/_241_quiz}



\end{document}